\documentclass[11pt, article, oneside]{memoir}

\settrims{0pt}{0pt} % page and stock same size
\settypeblocksize{*}{36pc}{*} % {height}{width}{ratio}
\setlrmargins{*}{*}{1} % {spine}{edge}{ratio}
\setulmarginsandblock{1in}{1in}{*} % height of typeblock computed
\setheadfoot{\onelineskip}{2\onelineskip} % {headheight}{footskip}
\setheaderspaces{*}{1.5\onelineskip}{*} % {headdrop}{headsep}{ratio}
\checkandfixthelayout



\usepackage{mathtools}
\usepackage{amsthm}
\usepackage{amssymb}
\usepackage{stmaryrd}
\usepackage{newpxtext}
\usepackage[utf8]{inputenc}
\usepackage[varg,bigdelims]{newpxmath}
\usepackage[usenames,dvipsnames]{xcolor}
\usepackage{tikz}
\usepackage{todonotes}
\usepackage{graphicx}
\usepackage{enumitem}
\usepackage{mathpartir}
\usepackage[bookmarks=true, colorlinks=true, linkcolor=blue!50!red, citecolor=orange,
pdfencoding=unicode]{hyperref}

\usepackage[capitalize]{cleveref}
  \newcommand{\creflastconjunction}{, and\nobreakspace}%Make cleveref use serial comma

\usepackage[backend=biber,style = alphabetic]{biblatex}
  \addbibresource{Library20171004.bib}



\usetikzlibrary{
	cd,
	math,
	decorations.markings,
	positioning,
	arrows.meta,
	shapes,
	calc,
	circuits.logic.US,
	fit,
	quotes,
	intersections}
\hypersetup{final}

  \tikzset{
     oriented WD/.style={%everything after equals replaces "oriented WD" in key.
        every to/.style={out=0,in=180,draw},
        label/.style={
           font=\everymath\expandafter{\the\everymath\scriptstyle},
           inner sep=0pt,
           node distance=2pt and -2pt},
        semithick,
        node distance=1 and 1,
        decoration={markings, mark=at position \stringdecpos with \stringdec},
        ar/.style={postaction={decorate}},
        execute at begin picture={\tikzset{
           x=\bbx, y=\bby,
           every fit/.style={inner xsep=\bbx, inner ysep=\bby}}}
        },
     string decoration/.store in=\stringdec,
     string decoration={\arrow{stealth};},
     string decoration pos/.store in=\stringdecpos,
     string decoration pos=.7,
     bbx/.store in=\bbx,
     bbx = 1.5cm,
     bby/.store in=\bby,
     bby = 1.5ex,
     bb port sep/.store in=\bbportsep,
     bb port sep=1.5,
     % bb wire sep/.store in=\bbwiresep,
     % bb wire sep=1.75ex,
     bb port length/.store in=\bbportlen,
     bb port length=4pt,
     bb penetrate/.store in=\bbpenetrate,
     bb penetrate=0,
     bb min width/.store in=\bbminwidth,
     bb min width=1cm,
     bb rounded corners/.store in=\bbcorners,
     bb rounded corners=2pt,
     bb small/.style={bb port sep=1, bb port length=2.5pt, bbx=.4cm, bb min width=.4cm, 
bby=.7ex},
		 bb medium/.style={bb port sep=1, bb port length=2.5pt, bbx=.4cm, bb min width=.4cm, 
bby=.9ex},
     bb/.code 2 args={%When you see this key, run the code below:
        \pgfmathsetlengthmacro{\bbheight}{\bbportsep * (max(#1,#2)+1) * \bby}
        \pgfkeysalso{draw,minimum height=\bbheight,minimum width=\bbminwidth,outer 
sep=0pt,
           rounded corners=\bbcorners,thick,
           prefix after command={\pgfextra{\let\fixname\tikzlastnode}},
           append after command={\pgfextra{\draw
              \ifnum #1=0{} \else foreach \i in {1,...,#1} {
                 ($(\fixname.north west)!{\i/(#1+1)}!(\fixname.south west)$) +(-
\bbportlen,0) 
  coordinate (\fixname_in\i) -- +(\bbpenetrate,0) coordinate (\fixname_in\i')}\fi 
  %Define the endpoints of tickmarks
              \ifnum #2=0{} \else foreach \i in {1,...,#2} {
                 ($(\fixname.north east)!{\i/(#2+1)}!(\fixname.south east)$) +(-
\bbpenetrate,0) 
  coordinate (\fixname_out\i') -- +(\bbportlen,0) coordinate (\fixname_out\i)}\fi;
           }}}
     },
     bb name/.style={append after command={\pgfextra{\node[anchor=north] at 
(\fixname.north) {#1};}}}
  }


  \tikzset{
  	unoriented WD/.style={
  		every to/.style={draw},
  		shorten <=-\penetration, shorten >=-\penetration,
  		label distance=-2pt,
  		thick,
  		node distance=\spacing,
  		execute at begin picture={\tikzset{
  			x=\spacing, y=\spacing}}
  		},
  	pack size/.store in=\psize,
  	pack size = 8pt,
  	spacing/.store in=\spacing,
  	spacing = 8pt,
  	link size/.store in=\lsize,
  	link size = 2pt,
		penetration/.store in=\penetration,
		penetration = 2pt,
  	pack color/.store in=\pcolor,
  	pack color = blue,
  	pack inside color/.store in=\picolor,
  	pack inside color=blue!20,
  	pack outside color/.store in=\pocolor,
  	pack outside color=blue!50!black,
  	surround sep/.store in=\ssep,
  	surround sep=8pt,
  	link/.style={
  		circle, 
  		draw=black, 
  		fill=black,
  		inner sep=0pt, 
  		minimum size=\lsize
  	},
  	pack/.style={
  		circle, 
  		draw = \pocolor, 
  		fill = \picolor,
  		inner sep = .25*\psize,
  		minimum size = \psize
  	},
  	outer pack/.style={
  		ellipse, 
  		draw,
  		inner sep=\ssep,
  		color=\pocolor,
  	},
  	intermediate pack/.style={
  		ellipse,
  		dashed, 
  		draw,
  		inner sep=\ssep,
  		color=\pocolor,
  	},
  }




\theoremstyle{plain}
\newtheorem{theorem}{Theorem}[chapter] %change [] to chapter if we want to change global numbering
\newtheorem{proposition}[theorem]{Proposition}
\newtheorem{corollary}[theorem]{Corollary}
\newtheorem{lemma}[theorem]{Lemma}
\newtheorem{conjecture}[theorem]{Conjecture}

\theoremstyle{definition}
\newtheorem{definition}[theorem]{Definition}
\newtheorem{construction}[theorem]{Construction}
\newtheorem{notation}[theorem]{Notation}
\newtheorem{axiom}{Axiom}
\newtheorem*{axiom*}{Axiom}

\theoremstyle{remark}
\newtheorem{example}[theorem]{Example}
\newtheorem{remark}[theorem]{Remark}
\newtheorem{warning}[theorem]{Warning}

\setcounter{axiom}{-1}

% Renewed commands

\renewcommand{\ss}{\subseteq}

% Macros %
\newcommand{\const}[1]{\mathtt{#1}}
\newcommand{\Set}[1]{\mathrm{#1}}
\newcommand{\cat}[1]{\mathcal{#1}}
\newcommand{\Cat}[1]{\mathbf{#1}}
\newcommand{\fun}[1]{\mathit{#1}}
\newcommand{\Fun}[1]{\mathsf{#1}}

\DeclareMathOperator{\id}{id}
\DeclareMathOperator{\Hom}{Hom}
\DeclareMathOperator{\Mor}{Mor}
\DeclareMathOperator*{\colim}{colim}
\DeclareMathOperator{\im}{im}
\DeclareMathOperator{\Ob}{Ob}
\DeclareMathOperator{\Typ}{Typ}

\newcommand{\op}{^\tn{op}}


\newcommand{\NN}{\mathbb{N}}

\newcommand{\cocolon}{:\!}
\newcommand{\iso}{\cong}
\newcommand{\To}[1]{\xrightarrow{#1}}
\newcommand{\Too}[1]{\xrightarrow{\;\;#1\;\;}}
\newcommand{\from}{\leftarrow}
\newcommand{\From}[1]{\xleftarrow{#1}}
\newcommand{\Fromm}[1]{\xleftarrow{\;\;#1\;\;}}
\newcommand{\surj}{\twoheadrightarrow}
\newcommand{\inj}{\rightarrowtail}

\newcommand{\tn}[1]{\textnormal{#1}}
\newcommand{\ol}[1]{\overline{#1}}
\newcommand{\ul}[1]{\underline{#1}}
\newcommand{\wt}[1]{\widetilde{#1}}

\newcommand{\Psh}[1]{\Fun{Psh}(#1)}

\newcommand{\Cospan}{\Cat{Cospan}}
\newcommand{\SmSet}{\Cat{Set}}
\newcommand{\Cob}{\Cat{Cob}}
\newcommand{\OO}{\cat{O}}

\newcommand{\lleg}[1]{#1^\backprime}
\newcommand{\rleg}[1]{#1^\prime}
\newcommand{\apex}[1]{\check{#1}}

\newcommand{\erase}[1]{}
\newcommand{\dtodo}[2][]{\todo[linecolor=white, backgroundcolor=white, bordercolor=gray, #1]{#2}}
\newcommand{\stodo}[2][]{\todo[color=red!30, #1]{#2}}



\setlist{nosep}
\linespread{1.1}
\allowdisplaybreaks
\setsecnumdepth{subsection}
\settocdepth{section}
\setlength{\parindent}{15pt}

%------------ Document ------------%
\begin{document}


\title{Wiring operads}

\author{
  Sophie Raynor
  \and 
  David I. Spivak\thanks{Spivak was supported by AFOSR grants 
FA9550--14--1--0031 and FA9550--17--1--0058.}
}
\date{}

\maketitle

%-------- Chapter --------%
\chapter{Introduction}

In this paper we introduce the notion of \emph{wiring operad}, a category-theoretic framework for organizing theories of composition, e.g.\ those that arise in various categorical structures.

%---- Section ----%
\section{Operads associated to various categorical structures}

There are many sorts of categorical structures---categories, monoidal categories, traced monoidal categories, hypergraph categories, operads, etc.---and each has an associated sort of wiring diagram. A category with said structure, call it $\cat{C}$, can ``interpret'' such a wiring diagram as a way of assembling new morphisms from old in $\cat{C}$. For example, the five sorts of categorical structures listed above can interpret the following sorts of wiring diagrams:%
\footnote{What we call wiring diagrams are strongly related to much more well-known concept called \emph{string diagrams}. The difference is that string diagrams do not include the exterior box, the composite, as an explicit part. In the work that follows, this difference is important, because the exterior box and the interior boxes have the same status: they are all simply objects in the wiring operad. We also prefer the term ``wiring'', because the term ``string'' also has other interpretations in the intersection of math and physics.}
\[
\begin{tikzpicture}
\begin{scope}[font=\footnotesize, text height=1.5ex, text depth=.5ex]
  \begin{scope}[oriented WD, bb port sep=1, bb port length=2.5pt, bb min width=.4cm, bby=.2cm, inner xsep=.2cm, x=.5cm, y=.3cm]
  	\node[bb={1}{1}] (Catf) {$f$};
  	\node[bb={1}{1}, right=1 of Catf] (Catg) {$g$};
  	\node[bb={0}{0}, fit=(Catf) (Catg)] (Cat) {};
  	\node[coordinate] at (Cat.west|-Catf_in1) (Cat_in1) {};
  	\node[coordinate] at (Cat.east|-Catg_out1) (Cat_out1) {};
  	\draw[shorten <=-2pt] (Cat_in1) -- (Catf_in1);
  	\draw (Catf_out1) -- (Catg_in1);
  	\draw[shorten >=-2pt] (Catg_out1) -- (Cat_out1);
  %
  	\node[bb={1}{2}, above right=-1.5 and 4 of Catf] (Monf) {$f$};
  	\node[bb={2}{1}, below right=-1 and 1 of Monf] (Mong) {$g$};
  	\node[bb={0}{0}, fit=(Monf) (Mong)] (Mon) {};
  	\node[coordinate] at (Mon.west|-Monf_in1) (Mon_in1) {};
  	\node[coordinate] at (Mon.west|-Mong_in2) (Mon_in2) {};
  	\node[coordinate] at (Mon.east|-Monf_out1) (Mon_out1) {};
  	\node[coordinate] at (Mon.east|-Mong_out1) (Mon_out2) {};
  	\draw[shorten <=-2pt] (Mon_in1) -- (Monf_in1);
  	\draw[shorten >=-2pt] (Monf_out1) -- (Mon_out1);
  	\draw (Monf_out2) to (Mong_in1);
  	\draw[shorten <=-2pt] (Mon_in2) -- (Mong_in2);
  	\draw[shorten >=-2pt] (Mong_out1) -- (Mon_out2);
  %
  	\node[bb={2}{1}, right= 4.5 of Monf] (Trf) {$f$};
  	\node[bb={2}{2}, below right=-1 and 1 of Trf] (Trg) {$g$};
  	\node[bb={0}{0}, fit={($(Trf.north west)+(-.5,1)$) ($(Trg.south east)+(.5,-1)$)}] (Tr) {};
  	\node[coordinate] at (Tr.west|-Trf_in2) (Tr_in1) {};
  	\node[coordinate] at (Tr.west|-Trg_in2) (Tr_in2) {};
  	\node[coordinate] at (Tr.east|-Trf_out1) (Tr_out1) {};
  	\node[coordinate] at (Tr.east|-Trg_out2) (Tr_out2) {};
  	\draw[shorten <=-2pt] (Tr_in1) -- (Trf_in2);
  	\draw (Trf_out1) to (Trg_in1);
  	\draw[shorten <=-2pt] (Tr_in2) -- (Trg_in2);
  	\draw[shorten >=-2pt] (Trg_out2) -- (Tr_out2);
  	\draw let \p1=(Trg.east), \p2=(Trf.north west), \n1=\bbportlen, \n2=\bby in
  		(Trg_out1) to[in=0] (\x1+\n1,\y2+\n2) -- (\x2-\n1,\y2+\n2) to[out=180] (Trf_in1);
  %
  \end{scope}
  \begin{scope}[penetration=0, unoriented WD, pack outside color=black, pack inside color=white]
  	\node[pack, right=2.9 of Trf] (Hypf) {$f$};
  	\node[pack, below right=0 and .5 of Hypf] (Hypg) {$g$};
  	\node[outer pack, inner sep=5pt, fit=(Hypf) (Hypg)] (Hyp) {};
  	\node[coordinate] at ($(Hypg.-30)!.5!(Hyp.-30)$) (link) {};
  	\draw (Hypf) to[bend left] (Hypg);
  	\draw (Hypf) to[bend right] (Hypg);
  	\draw (Hypg) -- (link);
  	\draw[shorten >= -2pt] (link) to[bend left] (Hyp.-20);
  	\draw[shorten >= -2pt] (link) to[bend right] (Hyp.-45);
  	\draw[shorten >= -2pt] (Hypf) -- (Hyp);
  \end{scope}
  \begin{scope}[circuit logic US, thick, every to/.style={out=0,in=180}]
  	\node[and gate, draw, right=2.5 of Hypf] (Opdf) {$f$};
  	\node[and gate, draw, below right=0 and 0.5 of Opdf] (Opdg) {$g$};
		\node[and gate, inner sep=1pt, draw, fit=(Opdf) (Opdg)] (Opd) {};
		\draw[shorten <=-2pt] (Opd.input 1|-Opdf.west) to (Opdf.west);
		\draw[shorten <=-2pt] (Opd.input 1|-Opdg.input 2) to (Opdg.input 2);
		\draw (Opdf.output) to (Opdg.input 1);
		\draw[shorten >=-2pt] (Opdg.output) to (Opd.output);
  \end{scope}
	\node[below=.65 of Cat.south] (Cat name) {category};
	\node[text width=1.5cm] at (Cat name-|Mon) {monoidal category};
	\node[text width=2.5cm] at (Cat name-|Tr) {traced monoidal category};
	\node[text width=2cm] at (Cat name-|Hyp) {hypergraph category};
	\node at (Cat name-|Opd) {operad};
\end{scope}
\end{tikzpicture}
\]
In each case, a morphism in $\cat{C}$ is being assembled from two morphisms $f,g\in\cat{C}$, which we might call \emph{components}. String diagrams in general can assemble a composite morphism from any number of component morphisms; for example the following build a composite morphism from three components:
\[
\begin{tikzpicture}[oriented WD, bbx = .3cm, bby =.3cm, bb min width=.5cm, bb port length=0, bb port sep=.8, font=\footnotesize, text height=1.5ex, text depth=.5ex]
	\node[bb={1}{1}] (X1) {$f$};
  	\node[bb={1}{1}, right=of X1] (X2) {$g$};
	\node[bb={1}{1}, right=of X2] (X3) {$h$};
	\node[bb={1}{1}, fit=(X1) (X2) (X3)] (Y) {};
	\draw[shorten <=-2pt] (Y_in1') to (X1_in1);
	\draw (X1_out1) to (X2_in1);
	\draw (X2_out1) to (X3_in1);
	\draw[shorten >=-2pt] (X3_out1) to (Y_out1');
%
	\node[bb={1}{2}, above right=-1 and 7 of X3] (A) {$f$};
	\node[bb={2}{2}, below right=-1 and 1 of A] (B) {$g$};
	\node[bb={2}{1}, above right=-1 and 1 of B] (C) {$h$};
	\node[bb={0}{0}, fit=(A) (B) (C)] (Y) {};
	\draw[shorten <=-2pt] (Y.west|-A_in1) to (A_in1);
	\draw[shorten <=-2pt] (Y.west|-B_in2) to (B_in2);
	\draw (A_out1) to (C_in1);
	\draw (A_out2) to (B_in1);
	\draw (B_out1) to (C_in2);	
	\draw[shorten >=-2pt] (C_out1) to (C_out1-|Y.east);
	\draw[shorten >=-2pt] (B_out2) to (B_out2-|Y.east);
\end{tikzpicture}
\]
The first composite can be denoted $h\circ g\circ f$; technically it should be either $(h\circ g)\circ f$ or $h\circ (g\circ f)$, but they are equal in any category. The wiring diagram ``abstracts away the difference''. The second composite is more complicated to write as a wiring of text---because it is inherently 2-dimensional compared to text which is 1-dimensional---but it can be denoted $(h\otimes\id)\circ(\id\otimes g)\circ(f\otimes\id)$.

Operads have been proposed as a way of organizing the various sorts of wiring diagrams \cite{Spivak:2013b,Rupel.Spivak:2013a}. An operad morphism has many inputs and one output, and these correspond to a wiring diagram assembling many component morphisms to make one composite morphism. For example, it was shown in \cite{Spivak.Schultz.Rupel:2016a} that the operad for traced monoidal categories is $\Cob$, the operad of oriented 1-dimensional cobordisms, and it was shown in \cite{Fong?} that the operad for hypergraph categories is $\Cospan$.

While the operad formalism neatly captures wiring composition---and can thus be used to define the above sorts of categorical structures as operad-algebras---there is a technical annoyance that emerges in each case: the operad handles morphisms, but not objects in these categorical structures. The objects must be ``baked in'' to the operad as a varying set of string labels. Thus it is not the case that traced monoidal categories are algebras on $\Cob$ as our informal statement in the previous paragraph may have suggested. Instead, it is the case that for any generating set $\Lambda$ of objects, traced-monoidal-categories-whose-object-set-is-generated-by-$\Lambda$ are algebras on $\Cob/\Lambda$. The latter is a mouthful, and is not entirely pleasing.

In this paper we remedy this by following an idea from the work of \cite{Joyal.Kock} and \cite{Raynor}. \stodo[inline]{Sophie, add details on the history of the monad on presheaves on elementary graphs idea}.


%---- Section ----%
\section{Notation and terminology}

For any natural number $n\in\NN=\{0,1,\ldots\}$, we denote the associated finite set by $\ul{n}\coloneqq\{1,2,\ldots,n\}$, so $\ul{0}=\varnothing$ and $\ul{3}=\{1,2,3\}$. Given a set $A$ and a function $a\colon\ul{n}\to A$, we may denote $a(i)$ by $a_i$ for $1\leq i\leq n$.

\begin{definition}\label{def.operad}
To specify an operad $\OO$,
\begin{itemize}
	\item One specifies a set $\Typ(\OO)$, elements of which will be called \emph{types}.
	\item For every natural number $n\in\NN$, function $x\colon\ul{n}\to\Typ(\OO)$, and type $y\in\Typ(\OO)$, one specifies a set $\OO(x;y)$, elements of which are called \emph{$n$-ary operations}. An $n$-ary operation $f\in\OO(x;y)$ may be denoted $f\colon(x_1,\ldots,x_n)\to y$, so a $0$-ary operation may be denoted $f\colon()\to y$.
	\item For types $x_1,\ldots,x_n,y\in\Typ(\OO)$ and bijection $\sigma\colon\ul{n}\To{\cong}\ul{n}$, one specifies a bijection $\OO(\sigma)\colon\OO(x;y)\To\cong\OO(x\sigma;y)$, called the \emph{symmetry}.
	\item For every type $x\in\Typ(\OO)$, one specifies a 1-ary operation $\id_x\colon(x)\to x$, called the \emph{identity on $x$}.
	\item For operations $g\colon(y_1,\ldots,y_n)\to z$ and $f_1\colon(x_{1,1},\ldots,x_{1,m_1})\to y_1$, $f_2\colon(x_{2,1},\ldots,x_{2,m_2})\to y_2$, \ldots $f_n\colon(x_{n,1},\ldots,x_{n,m_n})\to y_n$, one specifies an operation denoted $g\circ(f_1,\ldots,f_n)$, called the \emph{composite}.
\end{itemize}
These are required to satisfy well-known axioms; we refer the reader to \cite[Definition 2.2.21(?)]{Leinster:2004a}.
\end{definition}

\begin{remark}
What we call an operad is often called a ``small symmetric colored operad'' or a symmetric multicategory; a definition and plenty of examples can be found in \cite{Leinster:2004a}.

What we call types in \cref{def.operad} are often called colors or objects. What we call operations are often called morphisms.
\end{remark}
\todo[inline]{Do we use the symmetry in the proof? Can we drop it?}

%-------- Chapter --------%
\chapter{Wiring operads and associated categorical structures}

In this section we define wiring operads. A central character in this story will be $\Cospan$---the category of finite sets and cospans between them---which will turn out to be the terminal wiring operad. The sort of categorical structure associated to $\Cospan$ is that of hypergraph categories. We begin by recalling that story.

%---- Section ----%
\section{Hypergraph categories and $\Cospan$-algebras}

\begin{definition}
Let $(\Cospan,0,+)$ denote the symmetric monoidal category whose objects are natural numbers, and for which a morphism $\phi\colon m\to n$ consists of a finite set $\apex{\varphi}$, called the \emph{apex}, together with functions $\lleg\phi\colon\ul{m}\to\apex{\varphi}$ and $\rleg\phi\colon\ul{n}\to\apex{\varphi}$, called the left and right leg, respectively. The notation comes from the diagram
\[
\begin{tikzcd}
	\ul{m}\ar[dr, "\lleg\phi"']&&\ul{n}\ar[dl, "\rleg\phi"]\\
	&\apex{\phi}
\end{tikzcd}
\]
We consider two such cospans $\phi_1,\phi_2\colon \ul{m}\to \ul{n}$ to be the same morphism if there is a bijection $e\colon\apex{\phi_1}\to\apex{\phi_2}$ such that $e\lleg\phi_1=\lleg\phi_2$ and $e\rleg\phi_1=\rleg\phi_2$.

The composite of two cospans $\phi\colon m\to n$ and $\psi\colon n\to p$ is formed by pushout in the usual way, $\ul{m}\to(\apex{\phi}\sqcup_{\ul{n}}\apex{\psi})\from\ul{p}$, and the identity on $n$ is the pair of identities $\ul{n}\to \ul{n}\from \ul{n}$. The monoidal unit is $0$. The monoidal product of objects $m$ and $n$ is their sum $m+n$; the monoidal product of cospans $\phi_1\colon m_1\to n_1$ and $\phi_2\colon m_2\to n_2$ is the disjoint union across the board: $(\ul{m_1}+\ul{m_2})\to(\apex{\phi_1}+\apex{\phi_2})\from (\ul{n_1}+\ul{n_2})$.
\end{definition}

To any monoidal category $(\cat{M},I,\otimes)$ we can associate an operad $\OO_\cat{M}$, called the \emph{operad underlying $\cat{M}$}: it has $\Typ(\OO_\cat{M})\coloneqq\Ob(\cat{O})$ and operations $\OO_{\cat{M}}(x_1,\ldots,x_n;y)\coloneqq\cat{M}(x_1\otimes\cdots\otimes x_n,y)$; see \cite{Leinster:2004a}. We often blur the distinction between a monoidal category and its underlying operad; for example, we will denote $\OO_{\Cospan}$ simply by $\Cospan$.

We can picture a type $n$ in the operad $\Cospan$ as a circle with $n$ ports (in no particular order); here are pictures of the types $0,1,2,3$:
\[
\begin{tikzpicture}
[unoriented WD, 
spacing=10pt, pack size=12pt, surround sep=3pt, 
font=\tiny]
	\node[pack] (ff) {};
	\node[pack,right=3 of ff] (gg) {};
	\node[pack, right=3 of gg] (hh) {};
	\node[pack, right=3 of hh] (ii) {};
	\draw (gg) -- +(.5,0);
	\draw (hh) -- +(.5,0);
	\draw (hh) -- +(-.5,0);
	\draw (ii) -- +(-.2,.4);
	\draw (ii) -- +(-.2,-.4);
	\draw (ii) -- +(.5,0);
\end{tikzpicture}
\]

%-------- Chapter --------%
\chapter{Examples}


\printbibliography
\printindex


\end{document}