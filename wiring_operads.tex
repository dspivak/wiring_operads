\documentclass[11pt, article, oneside]{memoir}

\settrims{0pt}{0pt} % page and stock same size
\settypeblocksize{*}{36pc}{*} % {height}{width}{ratio}
\setlrmargins{*}{*}{1} % {spine}{edge}{ratio}
\setulmarginsandblock{1in}{1in}{*} % height of typeblock computed
\setheadfoot{\onelineskip}{2\onelineskip} % {headheight}{footskip}
\setheaderspaces{*}{1.5\onelineskip}{*} % {headdrop}{headsep}{ratio}
\checkandfixthelayout



\usepackage{mathtools}
\usepackage{amsthm}
\usepackage{amssymb}
\usepackage{stmaryrd}
\usepackage{newpxtext}
\usepackage[utf8]{inputenc}
\usepackage[varg,bigdelims]{newpxmath}
\usepackage[usenames,dvipsnames]{xcolor}
\usepackage{tikz}
\usepackage{todonotes}
\usepackage{graphicx}
\usepackage{enumitem}
\usepackage{mathpartir}
\usepackage[bookmarks=true, colorlinks=true, linkcolor=blue!50!red, citecolor=orange,
pdfencoding=unicode]{hyperref}
\usepackage[capitalize]{cleveref}
  \newcommand{\creflastconjunction}{, and\nobreakspace}%Make cleveref use serial comma

\usepackage[backend=biber,style = alphabetic]{biblatex}
  \addbibresource{Library20171004.bib}


% Add widecheck
\DeclareFontFamily{U}{mathx}{\hyphenchar\font45}
\DeclareFontShape{U}{mathx}{m}{n}{
      <5> <6> <7> <8> <9> <10>
      <10.95> <12> <14.4> <17.28> <20.74> <24.88>
      mathx10
      }{}
\DeclareSymbolFont{mathx}{U}{mathx}{m}{n}
\DeclareFontSubstitution{U}{mathx}{m}{n}
\DeclareMathAccent{\widecheck}{0}{mathx}{"71}
\DeclareMathAccent{\wideparen}{0}{mathx}{"75}

% tikz libraries
\usetikzlibrary{
	cd,
	math,
	decorations.markings,
	decorations.pathreplacing,
	positioning,
	arrows.meta,
	shapes,
	calc,
	circuits.logic.US,
	fit,
	quotes,
	intersections}
\hypersetup{final}

%drawing wiring diagrams
  \tikzset{
     oriented WD/.style={%everything after equals replaces "oriented WD" in key.
        every to/.style={out=0,in=180,draw},
        label/.style={
           font=\everymath\expandafter{\the\everymath\scriptstyle},
           inner sep=0pt,
           node distance=2pt and -2pt},
        semithick,
        node distance=1 and 1,
        decoration={markings, mark=at position \stringdecpos with \stringdec},
        ar/.style={postaction={decorate}},
        execute at begin picture={\tikzset{
           x=\bbx, y=\bby,
           every fit/.style={inner xsep=\bbx, inner ysep=\bby}}}
        },
     string decoration/.store in=\stringdec,
     string decoration={\arrow{stealth};},
     string decoration pos/.store in=\stringdecpos,
     string decoration pos=.7,
     bbx/.store in=\bbx,
     bbx = 1.5cm,
     bby/.store in=\bby,
     bby = 1.5ex,
     bb port sep/.store in=\bbportsep,
     bb port sep=1.5,
     % bb wire sep/.store in=\bbwiresep,
     % bb wire sep=1.75ex,
     bb port length/.store in=\bbportlen,
     bb port length=4pt,
     bb penetrate/.store in=\bbpenetrate,
     bb penetrate=0,
     bb min width/.store in=\bbminwidth,
     bb min width=1cm,
     bb rounded corners/.store in=\bbcorners,
     bb rounded corners=2pt,
     bb small/.style={bb port sep=1, bb port length=2.5pt, bbx=.4cm, bb min width=.4cm, 
bby=.7ex},
		 bb medium/.style={bb port sep=1, bb port length=2.5pt, bbx=.4cm, bb min width=.4cm, 
bby=.9ex},
     bb/.code 2 args={%When you see this key, run the code below:
        \pgfmathsetlengthmacro{\bbheight}{\bbportsep * (max(#1,#2)+1) * \bby}
        \pgfkeysalso{draw,minimum height=\bbheight,minimum width=\bbminwidth,outer 
sep=0pt,
           rounded corners=\bbcorners,thick,
           prefix after command={\pgfextra{\let\fixname\tikzlastnode}},
           append after command={\pgfextra{\draw
              \ifnum #1=0{} \else foreach \i in {1,...,#1} {
                 ($(\fixname.north west)!{\i/(#1+1)}!(\fixname.south west)$) +(-
\bbportlen,0) 
  coordinate (\fixname_in\i) -- +(\bbpenetrate,0) coordinate (\fixname_in\i')}\fi 
  %Define the endpoints of tickmarks
              \ifnum #2=0{} \else foreach \i in {1,...,#2} {
                 ($(\fixname.north east)!{\i/(#2+1)}!(\fixname.south east)$) +(-
\bbpenetrate,0) 
  coordinate (\fixname_out\i') -- +(\bbportlen,0) coordinate (\fixname_out\i)}\fi;
           }}}
     },
     bb name/.style={append after command={\pgfextra{\node[anchor=north] at 
(\fixname.north) {#1};}}}
  }


  \tikzset{
  	unoriented WD/.style={
  		every to/.style={draw},
  		shorten <=-\penetration, shorten >=-\penetration,
  		label distance=-2pt,
  		thick,
  		node distance=\spacing,
  		execute at begin picture={\tikzset{
  			x=\spacing, y=\spacing}}
  		},
  	pack size/.store in=\psize,
  	pack size = 8pt,
  	spacing/.store in=\spacing,
  	spacing = 8pt,
  	link size/.store in=\lsize,
  	link size = 2pt,
		penetration/.store in=\penetration,
		penetration = 2pt,
  	pack color/.store in=\pcolor,
  	pack color = blue,
  	pack inside color/.store in=\picolor,
  	pack inside color=blue!20,
  	pack outside color/.store in=\pocolor,
  	pack outside color=blue!50!black,
  	surround sep/.store in=\ssep,
  	surround sep=8pt,
  	link/.style={
  		circle, 
  		draw=black, 
  		fill=black,
  		inner sep=0pt, 
  		minimum size=\lsize
  	},
  	pack/.style={
  		circle, 
  		draw = \pocolor, 
  		fill = \picolor,
  		inner sep = .25*\psize,
  		minimum size = \psize
  	},
  	outer pack/.style={
  		ellipse, 
  		draw,
  		inner sep=\ssep,
  		color=\pocolor,
  	},
  	intermediate pack/.style={
  		ellipse,
  		dashed, 
  		draw,
  		inner sep=\ssep,
  		color=\pocolor,
  	},
  }




\theoremstyle{plain}
\newtheorem{theorem}{Theorem}[chapter] %change [] to chapter if we want to change global numbering
\newtheorem{proposition}[theorem]{Proposition}
\newtheorem{corollary}[theorem]{Corollary}
\newtheorem{lemma}[theorem]{Lemma}
\newtheorem{conjecture}[theorem]{Conjecture}

\theoremstyle{definition}
\newtheorem{definition}[theorem]{Definition}
\newtheorem{construction}[theorem]{Construction}
\newtheorem{notation}[theorem]{Notation}
\newtheorem{axiom}{Axiom}
\newtheorem*{axiom*}{Axiom}

\theoremstyle{remark}
\newtheorem{example}[theorem]{Example}
\newtheorem{remark}[theorem]{Remark}
\newtheorem{warning}[theorem]{Warning}

\setcounter{axiom}{-1}

% Renewed commands

\renewcommand{\ss}{\subseteq}

% Macros %
\newcommand{\const}[1]{\mathtt{#1}}
\newcommand{\Set}[1]{\mathrm{#1}}
\newcommand{\cat}[1]{\mathcal{#1}}
\newcommand{\Cat}[1]{\mathbf{#1}}
\newcommand{\fun}[1]{\mathit{#1}}
\newcommand{\Fun}[1]{\mathsf{#1}}

\DeclareMathOperator{\id}{id}
\DeclareMathOperator{\Hom}{Hom}
\DeclareMathOperator{\Mor}{Mor}
\DeclareMathOperator*{\colim}{colim}
\DeclareMathOperator{\im}{im}
\DeclareMathOperator{\Ob}{Ob}
\DeclareMathOperator{\Typ}{Typ}

\newcommand{\op}{^\tn{op}}


\newcommand{\NN}{\mathbb{N}}

\newcommand{\cocolon}{:\!}
\newcommand{\iso}{\cong}
\newcommand{\To}[1]{\xrightarrow{#1}}
\newcommand{\Too}[1]{\xrightarrow{\;\;#1\;\;}}
\newcommand{\from}{\leftarrow}
\newcommand{\From}[1]{\xleftarrow{#1}}
\newcommand{\Fromm}[1]{\xleftarrow{\;\;#1\;\;}}
\newcommand{\surj}{\twoheadrightarrow}
\newcommand{\inj}{\rightarrowtail}

\newcommand{\tn}[1]{\textnormal{#1}}
\newcommand{\ol}[1]{\overline{#1}}
\newcommand{\ul}[1]{\underline{#1}}
\newcommand{\wt}[1]{\widetilde{#1}}

\newcommand{\Psh}{\Fun{Psh}}

\newcommand{\Cospan}{\Cat{Cospan}}
\newcommand{\SmSet}{\Cat{Set}}
\newcommand{\Cob}{\Cat{Cob}}
\newcommand{\OO}{\cat{O}}

% Cospan stuff
\newcommand{\lleg}[1]{#1^\backprime}
\newcommand{\rleg}[1]{#1^\prime}
\newcommand{\apex}[1]{\widecheck{#1}}
\newcommand{\cmap}{c}

% Editing stuff
\newcommand{\erase}[1]{}
\newcommand{\dtodo}[2][]{\todo[linecolor=white, backgroundcolor=white, bordercolor=gray, #1]{#2}}
\newcommand{\stodo}[2][]{\todo[color=red!30, #1]{#2}}

%Sophie macros. Notation to be streamlined later
\newcommand\Gr{\Cat{Gr}}
\newcommand\El{\Cat{El}}
%\newcommand\g{\mathcal G}
%\newcommand\cc{\mathcal C}
%\newcommand{\esG}[1][\mathcal{G}]{\ensuremath{\mathsf{es}{(#1)}}}
%\newcommand{\elG}[1][\mathcal{G}]{\ensuremath{\mathsf{el}{(#1)}}}
\newcommand\stick{\shortmid}


\setlist{nosep}
\linespread{1.1}
\allowdisplaybreaks
\setsecnumdepth{subsection}
\settocdepth{section}
\setlength{\parindent}{15pt}

%------------ Document ------------%
\begin{document}


\title{Wiring operads}

\author{
  Sophie Raynor
  \and 
  David I. Spivak\thanks{Spivak was supported by AFOSR grants 
FA9550--14--1--0031 and FA9550--17--1--0058.}
}
\date{}

\maketitle

%-------- Chapter --------%
\chapter{Introduction}

In this paper we introduce the notion of \emph{wiring operad}, a category-theoretic framework for organizing theories of composition, e.g.\ those that arise in various categorical structures.

%---- Section ----%
\section{Operads associated to various categorical structures}

There are many sorts of categorical structures---categories, monoidal categories, traced monoidal categories, hypergraph categories, operads, etc.---and each has an associated sort of wiring diagram. A category with said structure, call it $\cat{C}$, can ``interpret'' such a wiring diagram as a way of assembling new morphisms from old in $\cat{C}$. For example, the five sorts of categorical structures listed above can interpret the following sorts of wiring diagrams:
\begin{equation}\label{eqn.some_WDs}
\begin{tikzpicture}
\begin{scope}[font=\footnotesize, text height=1.5ex, text depth=.5ex]
  \begin{scope}[oriented WD, bb port sep=1, bb port length=2.5pt, bb min width=.4cm, bby=.2cm, inner xsep=.2cm, x=.5cm, y=.3cm]
  	\node[bb={1}{1}] (Catf) {$f$};
  	\node[bb={1}{1}, right=1 of Catf] (Catg) {$g$};
  	\node[bb={0}{0}, fit=(Catf) (Catg)] (Cat) {};
  	\node[coordinate] at (Cat.west|-Catf_in1) (Cat_in1) {};
  	\node[coordinate] at (Cat.east|-Catg_out1) (Cat_out1) {};
  	\draw[shorten <=-2pt] (Cat_in1) -- (Catf_in1);
  	\draw (Catf_out1) -- (Catg_in1);
  	\draw[shorten >=-2pt] (Catg_out1) -- (Cat_out1);
  %
  	\node[bb={1}{2}, above right=-1.5 and 4 of Catf] (Monf) {$f$};
  	\node[bb={2}{1}, below right=-1 and 1 of Monf] (Mong) {$g$};
  	\node[bb={0}{0}, fit=(Monf) (Mong)] (Mon) {};
  	\node[coordinate] at (Mon.west|-Monf_in1) (Mon_in1) {};
  	\node[coordinate] at (Mon.west|-Mong_in2) (Mon_in2) {};
  	\node[coordinate] at (Mon.east|-Monf_out1) (Mon_out1) {};
  	\node[coordinate] at (Mon.east|-Mong_out1) (Mon_out2) {};
  	\draw[shorten <=-2pt] (Mon_in1) -- (Monf_in1);
  	\draw[shorten >=-2pt] (Monf_out1) -- (Mon_out1);
  	\draw (Monf_out2) to (Mong_in1);
  	\draw[shorten <=-2pt] (Mon_in2) -- (Mong_in2);
  	\draw[shorten >=-2pt] (Mong_out1) -- (Mon_out2);
  %
  	\node[bb={2}{1}, right= 4.5 of Monf] (Trf) {$f$};
  	\node[bb={2}{2}, below right=-1 and 1 of Trf] (Trg) {$g$};
  	\node[bb={0}{0}, fit={($(Trf.north west)+(-.5,1)$) ($(Trg.south east)+(.5,-1)$)}] (Tr) {};
  	\node[coordinate] at (Tr.west|-Trf_in2) (Tr_in1) {};
  	\node[coordinate] at (Tr.west|-Trg_in2) (Tr_in2) {};
  	\node[coordinate] at (Tr.east|-Trf_out1) (Tr_out1) {};
  	\node[coordinate] at (Tr.east|-Trg_out2) (Tr_out2) {};
  	\draw[shorten <=-2pt] (Tr_in1) -- (Trf_in2);
  	\draw (Trf_out1) to (Trg_in1);
  	\draw[shorten <=-2pt] (Tr_in2) -- (Trg_in2);
  	\draw[shorten >=-2pt] (Trg_out2) -- (Tr_out2);
  	\draw let \p1=(Trg.east), \p2=(Trf.north west), \n1=\bbportlen, \n2=\bby in
  		(Trg_out1) to[in=0] (\x1+\n1,\y2+\n2) -- (\x2-\n1,\y2+\n2) to[out=180] (Trf_in1);
  %
  \end{scope}
  \begin{scope}[penetration=0, unoriented WD, pack outside color=black, pack inside color=white]
  	\node[pack, right=2.9 of Trf] (Hypf) {$f$};
  	\node[pack, below right=0 and .5 of Hypf] (Hypg) {$g$};
  	\node[outer pack, inner sep=5pt, fit=(Hypf) (Hypg)] (Hyp) {};
  	\node[coordinate] at ($(Hypg.-30)!.5!(Hyp.-30)$) (link) {};
  	\draw (Hypf) to[bend left] (Hypg);
  	\draw (Hypf) to[bend right] (Hypg);
  	\draw (Hypg) -- (link);
  	\draw[shorten >= -2pt] (link) to[bend left] (Hyp.-20);
  	\draw[shorten >= -2pt] (link) to[bend right] (Hyp.-45);
  	\draw[shorten >= -2pt] (Hypf) -- (Hyp);
  \end{scope}
  \begin{scope}[circuit logic US, thick, every to/.style={out=0,in=180}]
  	\node[and gate, draw, right=2.5 of Hypf] (Opdf) {$f$};
  	\node[and gate, draw, below right=0 and 0.5 of Opdf] (Opdg) {$g$};
		\node[and gate, inner sep=1pt, draw, fit=(Opdf) (Opdg)] (Opd) {};
		\draw[shorten <=-2pt] (Opd.input 1|-Opdf.west) to (Opdf.west);
		\draw[shorten <=-2pt] (Opd.input 1|-Opdg.input 2) to (Opdg.input 2);
		\draw (Opdf.output) to (Opdg.input 1);
		\draw[shorten >=-2pt] (Opdg.output) to (Opd.output);
  \end{scope}
	\node[below=.65 of Cat.south] (Cat name) {category};
	\node[text width=1.5cm] at (Cat name-|Mon) {monoidal category};
	\node[text width=2.5cm] at (Cat name-|Tr) {traced monoidal category};
	\node[text width=2cm] at (Cat name-|Hyp) {hypergraph category};
	\node at (Cat name-|Opd) {operad};
\end{scope}
\end{tikzpicture}
\end{equation}
In each case, a morphism in $\cat{C}$ is being assembled from two morphisms $f,g\in\cat{C}$, which we might call \emph{components}. A given sort of wiring diagram can of course assemble a composite morphism from more than two component morphisms; for example the following build a composite morphism from three components:
\[
\begin{tikzpicture}[oriented WD, bbx = .3cm, bby =.3cm, bb min width=.5cm, bb port length=0, bb port sep=.8, font=\footnotesize, text height=1.5ex, text depth=.5ex]
	\node[bb={1}{1}] (X1) {$f$};
  	\node[bb={1}{1}, right=of X1] (X2) {$g$};
	\node[bb={1}{1}, right=of X2] (X3) {$h$};
	\node[bb={1}{1}, fit=(X1) (X2) (X3)] (Y) {};
	\draw[shorten <=-2pt] (Y_in1') to (X1_in1);
	\draw (X1_out1) to (X2_in1);
	\draw (X2_out1) to (X3_in1);
	\draw[shorten >=-2pt] (X3_out1) to (Y_out1');
%
	\node[bb={1}{2}, above right=-1 and 7 of X3] (A) {$f$};
	\node[bb={2}{2}, below right=-1 and 1 of A] (B) {$g$};
	\node[bb={2}{1}, above right=-1 and 1 of B] (C) {$h$};
	\node[bb={0}{0}, fit=(A) (B) (C)] (Y) {};
	\draw[shorten <=-2pt] (Y.west|-A_in1) to (A_in1);
	\draw[shorten <=-2pt] (Y.west|-B_in2) to (B_in2);
	\draw (A_out1) to (C_in1);
	\draw (A_out2) to (B_in1);
	\draw (B_out1) to (C_in2);	
	\draw[shorten >=-2pt] (C_out1) to (C_out1-|Y.east);
	\draw[shorten >=-2pt] (B_out2) to (B_out2-|Y.east);
\end{tikzpicture}
\]
The first composite can be denoted $h\circ g\circ f$; technically it should be either $(h\circ g)\circ f$ or $h\circ (g\circ f)$, but they are equal in any category. The wiring diagram ``abstracts away the difference''. The second composite is more complicated to write as a wiring of text---because it is inherently 2-dimensional compared to text which is 1-dimensional---but it can be denoted $(h\otimes\id)\circ(\id\otimes g)\circ(f\otimes\id)$.

Operads have been proposed as a way of organizing the various sorts of wiring diagrams \cite{Spivak:2013b,Rupel.Spivak:2013a}. An operad morphism has many inputs and one output, and these correspond to a wiring diagram assembling many component morphisms to make one composite morphism. For example, it was shown in \cite{Spivak.Schultz.Rupel:2016a} that the operad for traced monoidal categories is $\Cob$, the operad of oriented 1-dimensional cobordisms, and it was shown in \cite[Section 4.4.2]{Fong:2016a} that the operad for hypergraph categories is $\Cospan$.

This notion of wiring diagrams has a much better-known predecessor, namely the string diagram calculus as developed by Joyal, Street, and Verity in \cite{Joyal.Street:1993a, Joyal.Street.Verity:1996a}. We call our structures wiring diagrams to emphasize three minor but important differences from string diagrams:
\begin{enumerate}
	\item Whereas string diagrams are embedded topological spaces up to an equivalence relation called isotopy, wiring diagrams are combinatorial gadgets: morphisms are generally indexed by a set rather than a space.
	\item String diagrams are drawn with inner boxes but generally not an outer box. From an operadic point of view, the outer box is an essential aspect: it is the codomain of a morphism in the operad.
	\item String diagrams emphasize combinators, i.e.\ various ways of combining boxes in series, parallel, with feedback, etc. Wiring diagrams emphasize substitution of diagrams in for other diagrams, i.e.\ zooming and black-boxing.
\end{enumerate}

While the operadic approach to string/wiring diagrams neatly captures composition---and can thus be used to define the above sorts of categorical structures as operad-algebras---there is a technical annoyance that emerges. Namely, the operad handles morphisms but not objects in these categorical structures. The objects must be ``baked in'' to the operad as a varying set of string labels. Thus it is not the case that traced monoidal categories are algebras on $\Cob$ as our informal statement in the previous paragraph may have suggested. Instead, it is the case that for any generating set $\Lambda$ of objects, traced-monoidal-categories-whose-object-set-is-generated-by-$\Lambda$ are algebras on $\Cob/\Lambda$. The latter is a mouthful, and is not entirely pleasing.


In this paper we remedy this by following an idea from the work of \cite{Joyal.Kock} for defining compact symmetric multicategories (CSMs). Roughly, CSMs can be thought of as colored modular operads \cite{}, and they generalize small compact closed categories \cite{}, as well as colored wheeled properads \cite{}. It should be mentioned that the monad construction in \cite{Joyal.Kock} does not admit a well-defined multiplication. This is rectified in \cite{Raynor:2018a}. In \cite{}, an analogue construction for (non-wheeled) properads is presented, and it is clear how these methods may be further modified to handle other operad-types including (wheeled) PROPs \cite{} and (cyclic) operads \cite{}. 

We will not follow any of these constructions precisely, but instead take them as inspiration for our own approach. To construct their specific categorical structures of interest (CSMs), Joyal and Kock define a category $\El$ of \emph{elementary graphs}, as well as a certain monad on its category of presheaves; algebras on this monad are precisely the CSMs.

To explain in a bit more detail, the objects of $\El$ are connected Feynmann graphs without internal edges. Up to isomorphism, these are the \emph{stick graph} $(\stick)$ with no vertices, and for $n \geq 0$, the \emph{$n$-corolla} $C_n$ with one vertex and $n$ ``legs'':
\[
\begin{tikzpicture}[thick]
  \node[coordinate] (stick) at (0,0.25) {};
  \draw (stick) -- +(0,.5);
	\node[below=0.2 of stick] {$(\stick)$};
  \foreach \i in {0,...,4} {
  	\node[link] (C\i) at (\i+1,0.5) {};
		\ifnum \i>0 
  		\foreach \j in {1,...,\i} {
  			\draw (C\i) -- +(90+360*\j/\i:.5);
  		}
		\fi
  }
  \node[right= of C4] {$\cdots$};
\end{tikzpicture}
\]
For each leg $i$ of $C_n$, there is a morphism $c^n_i\colon(\stick)\to C_n$ in $\El$.

In our approach, the corollas will be replaced by circles with ports, which we call \emph{cells}, seen here:
\begin{equation}\label{eqn.cells}
\begin{tikzpicture}[thick]
  \node[coordinate] (stick) at (0,0.25) {};
  \draw (stick) -- +(0,.5);
	\node[below=0.2 of stick] {$(\stick)$};
  \foreach \i in {0,...,4} {
	\node[unoriented WD, pack inside color=white, pack outside color=black, pack] at (\i+1,0.5) (C\i) {};
		\ifnum \i>0 
  		\foreach \j in {1,...,\i} {
  			\draw (C\i) -- +(90+360*\j/\i:.5);
  		}
		\fi
  }
  \node[right= of C4] {$\cdots$};
\end{tikzpicture}
\end{equation}
These cells will eventually have the same role as the boxes and circles labeled $f$, $g$, etc.\ in \cref{eqn.some_WDs}, and the stick will have the same role as the ports and wires.

A presheaf $S\colon \El\op \to \SmSet$ can be roughly thought of as assigning preliminary data for a categorical structure $\cat{C}$. The set $S(\stick)$ will represent the objects of $\Ob(\cat{C})$ and for each $n\in\NN$, the set $S(C_n)$ will represent the morphisms of $\cat{C}$. The morphisms $c^n_i\colon(\stick)\to C_n$ will represent the domain/codomain of each morphism. To add a given sort of composition, one defines a monad on $\Psh(\El)$ whose algebras are to be the categorical structures with that sort of composition.

Making this precise is the goal of this paper, but we give a high-level view here. We will give a simple definition of what it means to equip an operad $\O$ with wiring structure: namely just a functor $\sigma\colon\O\to\Cospan$. We will use this to define both a small category $\El_\sigma$, of what we call elementary $\sigma$-graphs, together with a monad on $\Psh(\El_\sigma)$. Algebras on this monad will be categories that can interpret morphisms in $\O$---i.e.\ $\O$-shaped wiring diagrams---as composition laws.


%
% The stick graph is equipped with a non-trivial automorphism $\tau$ satisfying $\tau^2 = \id_{\stick}$. %, that changes the orientation of $(\stick)$. 
% Further, $\el$ contains all isomorphisms of corollas and for each $n$-corolla $\cc$, there are $2n$ morphisms $(\stick) \to \cc$ that can be pictured as inserting $ \stick$ into a leg of $\cc$ ``in either $\tau$-direction''. %To each finite set $X$, we may associate a corolla whose legs are labelled by $X$, and it is helpful to think of $\el $ as the groupoid of finite sets and isomorphisms, together with the extra object $ (\stick)$ and the morphisms with $(\stick)$ as their source.
 

%
%of $\el$ then consists of a set $S(\cc)$ for corolla $\cc$, together with the obvious symmetric group actions, and a set of \emph{types} (elsewhere called \emph{colours} or \emph{objects}) $\Lambda = S(\stick)$ equipped with an involution. % $ \omega = S(\tau)$. 
%For each leg of a corolla $\cc$, there are a pair of projections $ S(\cc) \to \Lambda$ that ``commute with the involution''.% and the image of each such pair corresponds to an orbit of $\omega$ in $\Lambda$.
%
%We can assemble a finite graph $\g$ with vertex set $V$ from a $V$-indexed set of corollas $\{\cc_v\}_{v \in V}$  ``glued together'' by stick graphs corresponding to the edges of $\g$. This gives a way of associating a unique subcategory $\esG \subset \el$ to any graph $\g$. Then for any graph $\g$, we may define \[
%S(\g) : = lim_{\cc \in \esG} S(\cc).\]
%Informally, $S(\g)$ is the set of ``decorations of $\g$ by $S$''. Each vertex $v$ of $ \g$ is decorated by an element of $\phi_v \in S(\cc_v)$ and, by construction, each edge of $\g$ is decorated by a type in $ \Lambda$ that is suitably ``compatible'' with each $\phi_v$ under the projections $S(\cc_v) \to \Lambda$.
%
%\stodo[inline]{Insert suitable graphic}
%
%Crucially for this paper, the object or type-labelling on the edges of a graph, just as the ``morphism or operation-labelling'' on the vertices, is simply part of the data of the presheaf. 

%The intention of \cite{Joyal.Kock} was to construct so-called \emph{Compact symmetric multicategories (CSMs)} as algebras for a monad defined on the category of presheaves on $\el$. Roughly, CSMs can be thought of as coloured modular operads \cite{}, and they generalise small compact closed categories \cite{}, as well as coloured wheeled properads \cite{}. It should be mentioned that the monad construction in \cite{Joyal.Kock} does not admit a well-defined multiplication. This is rectified in \cite{Raynor}. In \cite{}, an analogue construction for (non-wheeled) properads is presented, and it is clear how these methods may be further modified to handle other operad-types including (wheeled) PROPs \cite{} and (cyclic) operads \cite{}. 
%
%\stodo[inline]{Add references.}
%
%For the purposes of this article, the important point is that the monad for CSMs is constructed by evaluating $\el$ presheaves on graphs, as described above. CSMs are precisely those presheaves $S$ on $\el$ such that for any decoration of a connected graph $\g$ by $S$, there is a unique rule for ``collapsing'' the edges of $\g$ to get a single element (operation) in $S$. 
%  In particular, the categorical structures generalised by CSMs may be described by a single monad that deals with both the operations and the types at the same time. Furthermore, using the obvious modifications to the (elementary) graph category we may construct monads describing the other operad-like structures mentioned above, without having to worry about the particular set of types.

%---- Section ----%
\section{Notation and terminology}

For any natural number $n\in\NN=\{0,1,\ldots\}$, we denote the associated finite set by $\ul{n}\coloneqq\{1,2,\ldots,n\}$, so $\ul{0}=\varnothing$ and $\ul{3}=\{1,2,3\}$. Given a set $A$ and a function $a\colon\ul{n}\to A$, we may denote $a(i)\in A$ by $a_i$ for $1\leq i\leq n$.

\begin{definition}\label{def.operad}
To specify an operad $\OO$,
\begin{itemize}
	\item one specifies a set $\Typ(\OO)$, elements of which will be called \emph{types};
	\item for every natural number $k\in\NN$, function $x\colon\ul{k}\to\Typ(\OO)$, and type $y\in\Typ(\OO)$, one specifies a set $\OO(x;y)$, elements of which are called \emph{$k$-ary operations}. An $k$-ary operation $f\in\OO(x;y)$ may be denoted $f\colon(x_1,\ldots,x_k)\to y$, so a $0$-ary operation may be denoted $f\colon()\to y$;
	\item for types $x_1,\ldots,x_k,y\in\Typ(\OO)$ and bijection $\sigma\colon\ul{k}\To{\cong}\ul{k}$, one specifies a bijection $\OO(\sigma)\colon\OO(x;y)\To\cong\OO(x\sigma;y)$, called the \emph{symmetry};
	\item for every type $x\in\Typ(\OO)$, one specifies a 1-ary operation $\id_x\colon(x)\to x$, called the \emph{identity on $x$}; and
	\item for a $k$-ary operation $g\colon(y_1,\ldots,y_k)\to z$ and $k$-many morphisms $f_1\colon(x_{1,1},\ldots,x_{1,j_1})\to y_1$, $f_2\colon(x_{2,1},\ldots,x_{2,j_2})\to y_2$, \ldots, $f_k\colon(x_{k,1},\ldots,x_{k,j_k})\to y_k$, one specifies an operation denoted $g\circ(f_1,\ldots,f_k)$, called the \emph{composite}.
\end{itemize}
These are required to satisfy well-known axioms; we refer the reader to \cite[Definition 2.2.21(?)]{Leinster:2004a}.
\end{definition}

\begin{remark}
What we call an operad is often called a ``small symmetric colored operad'' or a symmetric multicategory; a definition and plenty of examples can be found in \cite{Leinster:2004a}.

What we call types in \cref{def.operad} are often called colors or objects. What we call operations are often called morphisms.
\end{remark}
\todo[inline]{Do we use the symmetry in the proof? Can we drop it?}

\begin{remark}\label{rem.mon_cat_operad_blur}
To any monoidal category $(\cat{M},I,\otimes)$ we can associate an operad $\OO_\cat{M}$, called the \emph{operad underlying $\cat{M}$}: it has $\Typ(\OO_\cat{M})\coloneqq\Ob(\cat{O})$ and operations $\OO_{\cat{M}}(x_1,\ldots,x_k;y)\coloneqq\cat{M}(x_1\otimes\cdots\otimes x_k,y)$; see \cite{Leinster:2004a}. This is part of a functor $U\colon\Cat{MonCatLax}\to\Cat{Opd}$, from the category of monoidal categories and lax monoidal functors to the category of operads and operad functors, and this functor is fully faithful. Thus we often blur the distinction between a monoidal category and its underlying operad
\end{remark}


%-------- Chapter --------%
\chapter{Wiring operads and associated categorical structures}

In this section we define wiring operads. A central character in this story will be $\Cospan$---the category of finite sets and cospans between them---which will turn out to be the terminal wiring operad. The sort of categorical structure associated to $\Cospan$ is that of hypergraph categories. We begin by recalling that story.

%---- Section ----%
\section{The operad $\Cospan$}

We first recall the definition of $\Cospan$ as a symmetric monoidal category, then discuss

\begin{definition}
Let $(\Cospan,0,+)$ denote the symmetric monoidal category whose objects are natural numbers, and for which a morphism $\cmap\colon m\to n$ consists of a finite set $\apex{\cmap}$, called the \emph{apex}, together with functions $\lleg\cmap\colon\ul{m}\to\apex{\cmap}$ and $\rleg\cmap\colon\ul{n}\to\apex{\cmap}$, called the left and right leg, respectively. The notation comes from the diagram
\[
\begin{tikzcd}
	\ul{m}\ar[dr, "\lleg\cmap"']&&\ul{n}\ar[dl, "\rleg\cmap"]\\
	&\apex{\cmap}
\end{tikzcd}
\]
We consider two such cospans $\cmap_1,\cmap_2\colon \ul{m}\to \ul{n}$ to be the same morphism if there is a bijection $e\colon\apex{\cmap_1}\to\apex{\cmap_2}$ such that $e\lleg\cmap_1=\lleg\cmap_2$ and $e\rleg\cmap_1=\rleg\cmap_2$.

The composite of two cospans $\cmap\colon m\to n$ and $\psi\colon n\to p$ is formed by pushout in the usual way, $\ul{m}\to(\apex{\cmap}\sqcup_{\ul{n}}\apex{\psi})\from\ul{p}$, and the identity on $n$ is the pair of identities $\ul{n}\to \ul{n}\from \ul{n}$. The monoidal unit is $0$. The monoidal product of objects $m$ and $n$ is their sum $m+n$; the monoidal product of cospans $\cmap_1\colon m_1\to n_1$ and $\cmap_2\colon m_2\to n_2$ is the disjoint union across the board: $(\ul{m_1}+\ul{m_2})\to(\apex{\cmap_1}+\apex{\cmap_2})\from (\ul{n_1}+\ul{n_2})$.
\end{definition}

By \cref{rem.mon_cat_operad_blur}, we can blur the distinction between $\Cospan$ as a monoidal category and $\Cospan$ as an operad. Thus by a minor abuse of notation, we will denote $\OO_{\Cospan}$ simply by $\Cospan$.

We can picture a type $n$ in the operad $\Cospan$ as a circle with $n$ ``ports'' placed around the circle in no particular order; here are pictures of the types $0,1,2,3,4$:
\[
\begin{tikzpicture}[thick]
  \foreach \i in {0,...,4} {
	\node[unoriented WD, pack inside color=white, pack outside color=black, pack] at (\i+1,0.5) (C\i) {};
		\ifnum \i>0 
  		\foreach \j in {1,...,\i} {
  			\draw (C\i) -- +(90+360*\j/\i:.5);
  		}
		\fi
  }
  \node[right= of C4] {$\cdots$};
\end{tikzpicture}
\]
A $k$-ary morphism $c\colon(m_1,\ldots,m_k)\to n$ in $\Cospan$ is a pair of functions
\[\ul{m_1+\cdots+m_k}\Too{\lleg\cmap}\apex{\cmap}\Fromm{\rleg\cmap}\ul{n}.\]
We draw it by putting a cell with $n$-ports on the outside, $k$-many cells with ports $m_1,\ldots,m_k$ on the inside, $\apex{c}$-many connection nodes in the space between, and connecting the various ports to the connection nodes according to the functions $\lleg\cmap$ and $\rleg\cmap$.

\begin{example}
The wiring diagram shown to the left below corresponds to the cospan $(4,2,3)\to 3$ shown to the right
\[
\begin{tikzpicture}[penetration=0, unoriented WD, spacing=20pt, pack size=25pt]
  \begin{scope}[font=\small]
  	\node[pack] (a) {$a$};
  	\node[pack, right=4 of a] (b) {$b$};
  	\node[pack] at ($(a)!.5!(b)+(0,-2)$) (c) {$c$};
  	\node[outer pack, fit=(a) (b) (c)] (outer) {};
  \end{scope}
%
	\begin{scope}[font=\fontsize{5pt}{0}\selectfont]
		\draw[shorten >= -3pt, shorten <=-3pt] (outer.200) to[bend left=50pt]
			node[pos=-.1] {$2$}
			node[pos=.5, link, label={[above right, font=\footnotesize]:$t$}] {}
			node[pos=1.1] {$1$}
			(outer.240);
		\draw[shorten >= -3pt] (a) to[bend left=20pt]
			node[pos=-.1] {$1$}
			node[pos=.5, link, label={[above, font=\footnotesize]:$u$}] {}
			node[pos=1.15] {$3$}
			(outer.170);
		\draw[shorten >= -3pt] (a) to[bend right=20pt]
			node[pos=-.15] {$2$}
			node[pos=.5, link, label={[right, font=\footnotesize]:$v$}] {}
			node[pos=1.25] {$4$}
			(outer.130);
  	\draw (a) to 
			node[pos=-.05] {$3$} 
			node[pos=.5, link, label={[above, font=\footnotesize]:$w$}] {} 
			node[pos=1.05] {$1$}
			(b);
		\draw (a) to[bend right]
			node[pos=-.1] {$4$}
			node[pos=.5, link, label={[above, font=\footnotesize]:$x$}] {}
			node[pos=1.1] {$1$}
			(c);
		\draw (b) to[bend left]
			node[pos=-.1] {$2$}
			node[pos=.5, link, label={[above, font=\footnotesize]:$y$}] (y) {}
			node[pos=1.1] {$2$}
			(c);
		\node[link, label={[above, font=\footnotesize]:$s$}] at ($(y)!.5!(outer.0)$) {};
		\draw[shorten >= -3pt] (c) to
			node[pos=-.15] {$3$}
			node[pos=.5, link, label={[left, font=\footnotesize]:$z$}] {}
			node[pos=1.25] {$6$}
			(outer);
		\draw[shorten >= -3pt] (y) to
			node[pos=1.15] {$5$}
			(outer.-30);
	\end{scope}
	\begin{scope}[font=\scriptsize,x=1em, decoration={brace, amplitude=5pt},y=3ex]
		\node[link, label=$1$, below right=0 and 10 of b.90] (a1) {};
		\node[link, label=$2$, right=1 of a1] (a2) {};
		\node[link, label=$3$, right=1 of a2] (a3) {};
		\node[link, label=$4$, right=1 of a3] (a4) {};
		\node[link, label=$1$, right=1 of a4] (b1) {};
		\node[link, label=$2$, right=1 of b1] (b2) {};
		\node[link, label=$1$, right=1 of b2] (c1) {};
		\node[link, label=$2$, right=1 of c1] (c2) {};
		\node[link, label=$3$, right=1 of c2] (c3) {};
		\draw[decorate] ($(a1.north)+(-2pt,10pt)$) to node[above=8pt] {$a$} ($(a4.north)+(2pt,10pt)$);
		\draw[decorate] ($(b1.north)+(-2pt,10pt)$) to node[above=8pt] {$b$} ($(b2.north)+(2pt,10pt)$);
		\draw[decorate] ($(c1.north)+(-2pt,10pt)$) to node[above=8pt] {$c$} ($(c3.north)+(2pt,10pt)$);
		\node[link, label=$s$] at ($(a1)!.5!(a2)+(0,-2.5)$) (s) {};
		\node[link, label=$t$, right=1 of s] (t) {};
		\node[link, label=$u$, right=1 of t] (u) {};
		\node[link, label=$v$, right=1 of u] (v) {};
		\node[link, label=$w$, right=1 of v] (w) {};
		\node[link, label=$x$, right=1 of w] (x) {};
		\node[link, label=$y$, right=1 of x] (y) {};
		\node[link, label=$z$, right=1 of y] (z) {};
		\node[link, label={[below]:$1$}] at ($(t)+(0,-2)$) (outer1) {};
		\node[link, label={[below]:$2$}, right=1 of outer1] (outer2) {};
		\node[link, label={[below]:$3$}, right=1 of outer2] (outer3) {};
		\node[link, label={[below]:$4$}, right=1 of outer3] (outer4) {};
		\node[link, label={[below]:$5$}, right=1 of outer4] (outer5) {};
		\node[link, label={[below]:$6$}, right=1 of outer5] (outer6) {};
	\end{scope}
	\begin{scope}[->, shorten >=8pt, thin, in=90, out=-90]
  	\draw (a1) to (u);
  	\draw (a2) to (v);
  	\draw (a3) to (w);
  	\draw (a4) to (x);
  	\draw (b1) to (w);
  	\draw (b2) to (y);
  	\draw (c1) to (x);
  	\draw (c2) to (y);
  	\draw (c3) to (z);
	\end{scope}
	\begin{scope}[->, shorten >=2pt, thin, in=-90, out=90]
		\draw (outer1) to (t);
		\draw (outer2) to (t);
		\draw (outer3) to (u);
		\draw (outer4) to (v);
		\draw (outer5) to (y);
		\draw (outer6) to (z);
	\end{scope}
\end{tikzpicture}
\]

\end{example}


%-------- Chapter --------%
\chapter{Examples}


\printbibliography
\printindex


\end{document}